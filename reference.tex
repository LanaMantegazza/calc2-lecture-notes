% Lana's reconstruction and subsequent ripoff of Asuka's lecture notes' LaTeX preamble.

\documentclass[12pt]{book}

% Page layout
\usepackage[
    letterpaper,
    reversemarginpar = true,
    asymmetric       = true,
    bindingoffset    = -0.5in,
    left             = 1.5in,
    right            = 1.25in,
    top              = 1.40in,
    bottom           = 1.8in,
    footskip         = .25in
]{geometry}

% Packages
\usepackage[T1]{fontenc}
\usepackage    {lmodern}
\usepackage    {amsmath,amssymb,amsthm}
\usepackage    {hyperref}
\usepackage    {calculator}
\usepackage    {etoolbox}
\usepackage    {parskip}
\usepackage    {geometry}


% Style for definitions, remarks, and examples: non-italicized text, bold title, numbered within sections
\newtheoremstyle{definitionstyle}
{1.0ex}     {1.0ex}
{}          {}
{\bfseries} {.}
{0.6em}     {}

% Definition of definitions, remarks, and examples
\theoremstyle{definitionstyle}
\newtheorem{definition}{Definition}[section]
\newtheorem{remark}[definition]{Remark}

% Definition of theorems, lemmas, examples, propositions, and corollaries
\theoremstyle{plain}
\newtheorem{theorem}[definition]{Theorem}
\newtheorem{lemma}[definition]{Lemma}
\newtheorem{example}[definition]{Example}
\newtheorem{proposition}[definition]{Proposition}
\newtheorem{corollary}[definition]{Corollary}

% Automatic numbering of equations
\numberwithin{equation}{section}

% Lengths and spacing
\makeatletter
\newlength{\boxmargin}
\setlength{\parskip}{0.375em}
\setlength{\boxmargin}{8pt}
\g@addto@macro\normalsize{
    \setlength\abovedisplayskip{12pt}
    \setlength\belowdisplayskip{5pt}
    \setlength\abovedisplayshortskip{0pt}
    \setlength\belowdisplayshortskip{5pt}
}
\newcommand{\midbox}[1]{
    \setlength{\fboxsep}{\boxmargin}
    \fbox{
	\parbox{\dimexpr\textwidth-(\boxmargin*2)\relax}{#1}
    }
}
\makeatother

% -------------------------------------------------------------------------

\begin{document}

\title{Lana's Better CALC II Lecture Notes}
\author{Lana Mantegazza}
\date{\today}
\maketitle
\tableofcontents

% Conditional chapter start reformatting
% \let\oldchapter\chapter
% \renewcommand{\chapter}[1]{\oldchapter{#1}\fontseries{l}\selectfont}
% \let\oldsection\sectioon
% \renewcommand{\section}[1]{\oldsection{#1}\textmd}

\chapter{Linear operators and diagonalization}

Let us assume $\mathbb{F}$ is always either the field $\mathbb{R}$ or the field
$\mathbb{C}$. We may use the word \textit{scalar} to mean ``element of
$\mathbb{F}$''. This chapter revolves around the concepts of eigenvalues and eigenvectors of
linear operators
\begin{equation}
    T : V \to V
\end{equation}
from a finite-dimensional vector space $V$ over $\mathbb{F}$ to itself, or
(equivalently) matrices
\begin{equation}
    A : \mathbb{F}^n \to \mathbb{F}^n.
\end{equation}

We shall also learn a procedure for ``diagonalising'' some square matrices,
which is of extreme importance in many applications. 

\midbox{\itshape Note that in Chapter~1 we
assume all matrices to be square and all linear operators\
to be from $V$ to $V$
(as opposed to going from $V$ to a different vector space $W$).}

\section{Linear Operators: Introduction and Review.}

We begin by recalling that for any $n$-dimensional $\mathbb{F}$-vector space
$V$, a choice of a basis $\mathcal{B} = \{ \mathbf{v}_1, \dots, \mathbf{v}_n \}$ determines an isomorphism $V \longrightarrow \mathbb{F}^n.$ Namely, the isomorphism is defined by
\begin{equation}
    \mathbf{v} = x_1 \mathbf{v}_1 + x_2 \mathbf{v}_2 + \cdots + x_n \mathbf{v}_n
    \longmapsto
    \begin{pmatrix}
	x_1 \\ x_2 \\ \vdots \\ x_n
    \end{pmatrix}.
\end{equation}

This is the map sending $\,\mathbf{v}\,$  to the column vector made up of the coefficients
$x_i \in \mathbb{F}$ of the unique representation of $\mathbf{v}$ as linear
combination $\mathbf{v} = x_1 \mathbf{v}_1 + x_2 \mathbf{v}_2 + \cdots + x_n \mathbf{v}_n$ in\linebreak the basis elements.

This chapter is concerned with linear operators on a vector space $V$.

\begin{definition}
    Let $V$ be a vector space. A linear transformation $T : V \to V$
    is\linebreak called a \textbf{linear operator} on $\,V\,$. The set of linear operators on $V$
    is denoted $\operatorname{End}(V)$.\footnote{Note that linear operators on $V$ would, in more general categorical language, be called \textit{endomorphisms} of $V$, hence the notation $\operatorname{End}(V)$.}
\end{definition}

Linear transformations $T : \mathbb{F}^n \to \mathbb{F}^n$ can be represented by $n \times n$ matrices, as was explained in
Linear Algebra and Geometry I (LAG-I). We introduce the following
notation.

\begin{definition}
    Let $M_n(\mathbb{F})$ denote the set of $n \times n$ matrices with entries
    in $\mathbb{F}$.\linebreak To summarize:
\end{definition}

testesnuts

\midbox{
    \setlength{\parskip}{1.1em}
    Suppose $V$ is an $n$-dimensional vector space over $\mathbb{F}$
    \textit{with a fixed basis} $\mathcal{B}$. Then
    \begin{itemize}
	\setlength\itemsep{-0.25em}
	\item[\Large \textbullet] $V$ can be identified with $\mathbb{F}^n$ by the isomorphism described in~\eqref{eq:1.1.1}, and
	\item[\Large \textbullet] the set $\operatorname{End}(V)$ of linear operators on $V$ is identified with the set $M_n(\mathbb{F})$.
    \end{itemize}


    Both of the identifications above depend on the choice of basis $\mathcal{B}$.
}

\begin{example}
    Recall that $
    A =
    \begin{pmatrix}
	a & b \\
	c & d
    \end{pmatrix}
    \in M_2(\mathbb{F})
    $ 
    represents the linear operator on $\mathbb{F}^2$,
    \[
	\begin{pmatrix}
	    x_1 \\ x_2
	\end{pmatrix}
	\longmapsto
	\begin{pmatrix}
	    a & b \\
	    c & d
	\end{pmatrix}
	\begin{pmatrix}
	    x_1 \\ x_2
	\end{pmatrix}
	=
	\begin{pmatrix}
	    a x_1 + b x_2 \\
	    c x_1 + d x_2
	\end{pmatrix}.
	\tag{1.1.2}
    \]

    If $V$ is a $2$-dimensional vector space over $\mathbb{F}$ with basis
    $\mathcal{B} = \{ v_1, v_2 \}$, then we have an isomorphism
    $
    V \longrightarrow \mathbb{F}^2
    $
    defined as in~\eqref{eq:1.1.1} that sends
    \[
	\mathbf{v} = x_1 \mathbf{v}_1 + x_2 \mathbf{v}_2
	\longmapsto
	\begin{pmatrix}
	    x_1 \\ x_2
	\end{pmatrix}.
	\tag{1.1.3}
    \]

    The matrix $A$ therefore determines a linear operator $T$ on $V$ sending $ \mathbf{v} = x_1 \mathbf{v}_1 + x_2 \mathbf{v}_2 $ \linebreak to the vector
    \[
	T(\mathbf{v}) = (a x_1 + b x_2) \mathbf{v}_1 + (c x_1 + d x_2) \mathbf{v}_2.
	\tag{1.1.4}
    \]

    This construction, which turns $A \in M_2(\mathbb{F})$ into the linear operator
    $T \in \operatorname{End}(V)$, describes the identification between
    $M_2(\mathbb{F})$ and $\operatorname{End}(V)$. Observe how the definition of the
    linear operator $T : V \to V$ from the matrix $A$ relies on the choice of basis
    $\mathcal{B}$.
\end{example}

\begin{remark}
\end{remark}


	\end{document}
